\documentclass{article}
\pagenumbering{arabic}
\usepackage{graphicx,amsmath,amssymb,bm,tikz}
\usetikzlibrary{calc,patterns,decorations.pathmorphing,decorations.markings}
\usepackage{xfrac}
\usepackage{accents}
\usepackage[utf8]{inputenc}
\usepackage{hyperref}
% Format fref 
\usepackage[plain,english]{fancyref}
\usepackage[margin=1in]{geometry} 
\fancyrefaddcaptions{english}{\renewcommand*{\frefeqname}{Eq.}}
% Figure Packages
\usepackage[outercaption]{sidecap}
\usepackage[export]{adjustbox}
\usepackage{graphicx}
\usepackage{caption}
\usepackage{wrapfig}
\usepackage{float}
\usepackage{algorithm, algorithmic, amsfonts,amsmath,amssymb,amsthm, color,comment,enumitem, environ, fancyhdr,   graphicx, mathtools, wasysym}
\pagestyle{fancy}
\setlength{\headheight}{22.5pt}
\newenvironment{problem}[2][Problem]{\begin{trivlist}
\item[\hskip \labelsep {\bfseries #1}\hskip \labelsep {\bfseries #2.}]}{\end{trivlist}}
\newenvironment{sol}
    {\emph{Solution:}
    }
    {
    \qed
    }
\specialcomment{com}{ \color{blue} \textbf{Comment:} }{\color{black}} %for instructor comments while grading
\NewEnviron{probscore}{\marginpar{ \color{blue} \tiny Problem Score: \BODY \color{black} }}
%%%%%%%%%%%%%%%%%%%%%%%%%%%%%%%%%%%%%%%%%%%%%%%%%%%%%%%%%%%%%%%%%%%%%%%%%%%%%%%%%





%%%%%%%%%%%%%%%%%%%%%%%%%%%%%%%%%%%%%%%%%%%%%
%Fill in the appropriate information below
\lhead{Daniel Agramonte}  %replace with your name
\rhead{MCHE 8380 \\ Homework 1} %replace XYZ with the homework course number, semester (e.g. ``Spring 2019"), and assignment number.
%%%%%%%%%%%%%%%%%%%%%%%%%%%%%%%%%%%%%%%%%%%%%



% Table stuff
\usepackage{multirow}
%\usepackage{floatrow}
%	\floatsetup[table]{capposition=top}%puts table caption above
% Change \subsection title characteristics
    \usepackage[parfill]{parskip}   % forces parskip to not affect headings
    \usepackage{enumitem}           % used for editing itemize environment
    \usepackage{titlesec}
        \titleformat*{\section}{\Large\bfseries\titlerule\vspace{0.5em}}
% Quote blocks
    \usepackage{csquotes} % use environment 'displayquote'
% Misc document settings
    \title{\Huge MCHE 8380 Homework 1} \author{Daniel Agramonte} \date{02.11.21}
    \setlength{\parindent}{0pt}
    \setlength{\parskip}{1em}
    \setlist{nosep, itemsep=0pt, parsep=0pt}
% Misc vocab commands
    \newcommand{\msalg}{{\fontfamily{cmtt}\selectfont ms83}}
    \newcommand{\lsq}{\emph{lsqnonlin}}
    \newcommand{\msalge}{{\fontfamily{cmtt}\selectfont MCHE\_6500\_NIST\_POLY}}
%
% MATLAB packages
%
\usepackage[framed,numbered]{matlab-prettifier}
\usepackage{textcomp}
\usepackage{listings}
%
% Matrix Spacing
%
\makeatletter
\renewcommand*\env@matrix[1][\arraystretch]{%
  \edef\arraystretch{#1}%
  \hskip -\arraycolsep
  \let\@ifnextchar\new@ifnextchar
  \array{*\c@MaxMatrixCols c}}
\makeatother
%
\begin{document}
\maketitle

\begin{enumerate}
    \item The vector $\boldsymbol{r} = x\boldsymbol{e}_{x}+y\boldsymbol{e}_{y}+z\boldsymbol{e}_{z}$ is referred to as the \textit{position vector} in Cartesian coordinates. Find $\nabla\cdot\boldsymbol{r}$ and $\nabla\times\boldsymbol{r}$.
    \\
    \\
    \begin{sol}
    We may treat $\nabla$ as the following vector:
    \begin{flalign*}
        \nabla &= \frac{\partial}{\partial x}\boldsymbol{e}_{x}+\frac{\partial}{\partial y}\boldsymbol{e}_{y}+\frac{\partial}{\partial z}\boldsymbol{e}_{z}.&&
    \end{flalign*}
    \end{sol}
    \item Show that if $S_{ij}$ is symmetric and $A_{ij}$ is antisymmetric, then $S_{ij}A_{ij}$. Express this result in matrix notation.
    \\
    \\
    \begin{sol}
    Since $S_{ij}$ is symmetric and $A_{ij}$ is antisymmetric, $S_{ij}A_{ij}=c=-S_{ji}A_{ji}=-c^{T}$, but since $c$ is a scalar, $c=c^{T} \therefore -c^{T}=-c=c \therefore 2c=0 \therefore c=S_{ij}A_{ij}=0$, as required. 
    \\
    \\
    We may express this result in matrix notation as follows: $\underaccent{\tilde}{\boldsymbol{S}}:\underaccent{\tilde}{\boldsymbol{A}}=0$
    \\
    \end{sol}
    \item Substitute $u_{i}=B_{ij}v_{j}$ and $C_{ij}=p_{i}q_{j}$ into $w_{i}=C_{ij}u_{j}$, and transcribe the result into direction (vector-operator) notation.
    \\
    \\
    \begin{sol}
    \begin{flalign*}
        w_{ij} &= p_{i}q_{j}B_{jk}v_{j}&& \\
        \underaccent{\tilde}{\boldsymbol{W}} &= \boldsymbol{p}\boldsymbol{q}^{T}\underaccent{\tilde}{\boldsymbol{B}}^{T}\boldsymbol{v}
    \end{flalign*}
    \end{sol}
    \item 
\end{enumerate}

\end{document}
