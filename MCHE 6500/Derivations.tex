\documentclass{article}
\usepackage[utf8]{inputenc}
\usepackage[fleqn]{amsmath}
\usepackage{amssymb}
\usepackage{hyperref}
\usepackage[plain,english]{fancyref}

% Make v cross
\newcommand{\velocity}{\mathop{\ooalign{\hfil$v$\hfil\cr\kern0.08em--\hfil\cr}}\nolimits}

\newcommand{\volsym}{\rlap{\kern.08em--}V}
\newcommand{\volsubsym}{\rlap{\scriptsize\kern.08em--}V}

\newcommand{\dd}{\mathrm{d}}

%Format fref 
\fancyrefaddcaptions{english}{%
  \renewcommand*{\frefeqname}{Eq.}%
}

% Custom enumeration scheme
\newcommand\setItemnumber[1]{\setcounter{enumi}{\numexpr#1-1\relax}}

% Format margins
\addtolength{\oddsidemargin}{-.875in}
\addtolength{\evensidemargin}{-.875in}
\addtolength{\textwidth}{1.75in}
\addtolength{\topmargin}{-.875in}
\addtolength{\textheight}{1.75in}

\title{4500/6500 Lecture 8 and 9 Equations and Derivations}
\author{Daniel Agramonte}
\date{September 22 2020}

\begin{document}

\maketitle

\section{Stagnation Functions}
\subsection{Stagnation Enthalpy}
\noindent For Stagnation Enthalpy we're given the following relation:
\begin{flalign}
    h_{0}&=h+\frac{\velocity^{2}}{2}.&& \label{eq:stag_enthalpy}
\end{flalign}

\subsection{Stagnation Temperature}
\noindent Recall that
\begin{flalign}
    R &= C_{p}-C_{v}&& \nonumber
\end{flalign}
\noindent and
\begin{flalign}
    \gamma &= \frac{C_{p}}{C_{v}}&& \nonumber \\
    \therefore C_{v} &= \frac{C_{p}}{\gamma}.&& \nonumber 
\end{flalign}
\noindent Substituting,
\begin{flalign}
    R &= C_{p}\left(1-\frac{1}{\gamma}\right)=C_{p}\left(\frac{\gamma-1}{\gamma}\right)&& \nonumber \\
    \therefore C_{p} &= \frac{R\gamma}{\gamma-1}&& \nonumber
\end{flalign}
\noindent Noting that for small changes in temperature,
\begin{flalign}
    \Delta h &= C_{p}\Delta T.&& \nonumber
\end{flalign}
\noindent We note that
\begin{flalign}
    h_{0}-h &= C_{p}(T_{0}-T)=\frac{\velocity^{2}}{2}&& \nonumber \\
    T_{0}-T&=\frac{\velocity^{2}}{2C_{p}}&& \nonumber \\
    T_{0}&=T+\frac{\velocity^{2}}{2C_{p}}&& \nonumber \\
    T_{0}&=T\left(1+\frac{\velocity^{2}}{2C_{p}T}\right).&& \nonumber
\end{flalign}
\noindent Substituting,
\begin{flalign}
    T_{0}&=T\left(1+\frac{\left(\gamma-1\right)\velocity^{2}}{2\gamma RT}\right).&& \nonumber  
\end{flalign}
\noindent Recalling that
\begin{flalign}
    M&\equiv \frac{\velocity}{a}=\frac{\velocity}{\sqrt{\gamma RT}},&& \nonumber
\end{flalign}
\noindent we can simplify our expression as follows
\begin{flalign}
    \frac{T_{0}}{T}&=\left(1+\frac{\left(\gamma-1\right)\velocity^{2}}{2\gamma RT}\right)=\left(1+\left(\frac{\gamma-1}{2}\right)\left(\frac{\velocity^{2}}{\gamma RT}\right)\right)=1+\frac{(\gamma-1)M^{2}}{2}&& \nonumber \\
    \therefore \frac{T}{T_{0}} &= \frac{2}{2+(\gamma-1)M^{2}}.&& \label{eq:stag_temp}
\end{flalign}
\noindent Solving for $M$,
\begin{flalign}
    \frac{T_{0}}{T}&=1+\frac{(\gamma-1)M^{2}}{2}&& \nonumber \\
    2\left(\frac{T_{0}}{T}-1\right)&=(\gamma-1)M^{2}&& \nonumber \\
    M^{2}&=\frac{2\left(\textstyle{\frac{T_{0}}{T}}-1\right)}{\gamma-1}&& \nonumber \\
    M&=\left(\frac{2\left(\textstyle{\frac{T_{0}}{T}}-1\right)}{\gamma-1}\right)^{0.5}.&& \label{inverse_stag_temp}
\end{flalign}
\subsection{Stagnation Pressure}
\noindent Recall the following isentropic relationship,
\begin{flalign}
    \left(\frac{P_{2}}{P_{1}}\right)_{s}&=\left(\frac{T_{2}}{T_{1}}\right)^{\textstyle{\frac{\gamma}{\gamma-1}}}.&& \nonumber
\end{flalign}
\noindent Letting state 1 be any point along a stream line, and state 2 be the flow if it were to be isentropically brought to rest,
\begin{flalign}
    \frac{P_{0}}{P}&=\left(\frac{T_{0}}{T}\right)^{\textstyle{\frac{\gamma}{\gamma-1}}}.&& \nonumber
\end{flalign}
\noindent Substituting in our previous expression,
\begin{flalign}
    \frac{P_{0}}{P}&=\left(1+\frac{(\gamma-1)M^{2}}{2}\right)^{\textstyle{\frac{\gamma}{\gamma-1}}}&& \nonumber \\
    \frac{P}{P_{0}}&=\left(1+\frac{(\gamma-1)M^{2}}{2}\right)^{\textstyle{\frac{\gamma}{1-\gamma}}}.&& \label{eq:stag_pressure}
\end{flalign}
\noindent Solving for $M$,
\begin{flalign}
    \left(\frac{P_{0}}{P}\right)^{\textstyle{\frac{\gamma-1}{\gamma}}}&=1+\frac{(\gamma-1)M^{2}}{2}&& \nonumber\\
    2\left(\left(\frac{P_{0}}{P}\right)^{\textstyle{\frac{\gamma-1}{\gamma}}}-1\right)&=(\gamma-1)M^{2}&& \nonumber\\
    M^{2}&=2\left(\displaystyle{\left(\frac{P_{0}}{P}\right)}^{\textstyle{\frac{\gamma-1}{\gamma}}}-1\right)(\gamma-1)^{-1}&& \nonumber\\
    M&=\left(2\left(\displaystyle{\left(\frac{P_{0}}{P}\right)}^{\textstyle{\frac{\gamma-1}{\gamma}}}-1\right)(\gamma-1)^{-1}\right)^{0.5}.&& \label{eq:inverse_stag_pressure}
\end{flalign}
\newpage
\subsection{Stagnation Density}

\noindent This is the part I feel least comfortable with. The derivation I see most common online starts with the following equation (I'm not sure where it comes from):
\begin{flalign}
    \left(\frac{\rho_{2}}{\rho_{1}}\right)_{s}&=\left(\frac{T_{2}}{T_{1}}\right)^{\textstyle{\frac{1}{\gamma-1}}}.&& \nonumber
\end{flalign}
\noindent Letting state 1 be any point along a stream line, and state 2 be the flow if it were to be isentropically brought to rest,
\begin{flalign}
    \frac{\rho_{0}}{\rho}&=\left(\frac{T_{0}}{T}\right)^{\textstyle{\frac{1}{\gamma-1}}}.&& \nonumber
\end{flalign}
\noindent Substituting in our previous expression,
\begin{flalign}
    \frac{\rho_{0}}{\rho}&=\left(1+\frac{(\gamma-1)M^{2}}{2}\right)^{\textstyle{\frac{1}{\gamma-1}}}&& \nonumber \\
    \frac{\rho}{\rho_{0}}&=\left(1+\frac{(\gamma-1)M^{2}}{2}\right)^{\textstyle{\frac{1}{1-\gamma}}}.&& \label{eq:stag_density}
\end{flalign}
\noindent Solving for $M$,
\begin{flalign}
    \left(\frac{\rho_{0}}{\rho}\right)^{\gamma-1}&=\left(1+\frac{(\gamma-1)M^{2}}{2}\right)&& \nonumber \\
    2\left(\left(\frac{\rho_{0}}{\rho}\right)^{\gamma-1}-1\right)&=(\gamma-1)M^{2}&& \nonumber \\
    M^{2}&=2\left(\left(\frac{\rho_{0}}{\rho}\right)^{\gamma-1}-1\right)(\gamma-1)^{-1}&& \nonumber \\
    M&=\left(2\left(\left(\frac{\rho_{0}}{\rho}\right)^{\gamma-1}-1\right)(\gamma-1)^{-1}\right)^{0.5}.&& \label{eq:stag_density_inverse}
\end{flalign}
\section{Area Variation Effects in Isentropic Compressible Flow}
\noindent Recall that for steady-state,
\begin{flalign}
    \dot{m}&=\rho\velocity A = \text{constant}&& \nonumber \\
    \rho &= \frac{P}{RT}.&& \nonumber
\end{flalign}
\noindent The flow condition when $M=1$ is known as \textit{sonic flow}. In general,
\begin{flalign}
    M&=\frac{\velocity}{a}=\frac{\velocity}{\sqrt{\gamma RT}}&& \nonumber \\
    \velocity&= M\sqrt{\gamma RT}.&& \nonumber
\end{flalign}
\noindent Substituting in \fref{eq:stag_temp} and \fref{eq:stag_pressure},
\begin{flalign}
    \dot{m}&=\frac{P}{RT}M\sqrt{\gamma RT} A = \text{constant}&& \nonumber \\
    \dot{m}&=P_{0}\left(1+\frac{(\gamma-1)M^{2}}{2}\right)^{\textstyle{\frac{-\gamma}{\gamma-1}}}T_{0}^{-1}\left(1+\frac{(\gamma-1)M^{2}}{2}\right)R^{-1}M\sqrt{\gamma RT} A = \text{constant}&& \nonumber \\
\end{flalign}
\end{document}