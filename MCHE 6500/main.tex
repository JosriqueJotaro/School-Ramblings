\documentclass{article}
\usepackage[utf8]{inputenc}
\usepackage[fleqn]{amsmath}

% Format margins
\addtolength{\oddsidemargin}{-.875in}
\addtolength{\evensidemargin}{-.875in}
\addtolength{\textwidth}{1.75in}
\addtolength{\topmargin}{-.875in}
\addtolength{\textheight}{1.75in}

\title{4500/6500 Lecture 4 Questions}
\author{Daniel Agramonte}
\date{September 06 2020}

\begin{document}

\maketitle

\section{Questions}
\begin{enumerate}
    \item When you say that 
        \begin{flalign}
            \delta W_{c} &= \delta W_{rev.}+\delta W_{sys.} = \delta Q_{R} - \text{d}E_{c}, 
        \end{flalign}
        could you explain a little more how you get this from the first law? I understand every term in the equation except for the $\text{d}E_{c}$ term. Also, tangentially, are we implying that these terms have a dot above them, e.g. $\delta \dot{Q}_{R}$, $\text{d}\dot{E}_{c}$, etc.?
    \item Similarly, a very important step that we make is showing that 
        \begin{flalign}
            \frac{\delta Q_{R}}{\delta T_{R}}&=\frac{\delta Q}{\delta T}.
        \end{flalign}
        I'm still not exactly sure where this comes from; I believe you mentioned that it had something to do with the reversibility of the processes inside the system? Could you elaborate some more about this?
    \item And my last question, when you say that, for a single cycle,
        \begin{flalign}
            \oint \text{d}E_{c} &= 0,
        \end{flalign}
        do you mean that the area enclosed by the cycle remains constant, and because of this, the cyclic integral is 0? Am I understanding that correctly?
\end{enumerate}

\end{document}
