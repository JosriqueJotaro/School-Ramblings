\documentclass{article}
\usepackage[utf8]{inputenc}
\usepackage[fleqn]{amsmath}
\usepackage{amssymb}
\usepackage{hyperref}
%
% MATLAB packages
%
\usepackage[framed,numbered]{matlab-prettifier}
\usepackage{textcomp}
\usepackage{xcolor}
\usepackage{listings}

% Make v cross
\makeatletter
\DeclareRobustCommand{\volume}{\text{\volumedash}V}
\newcommand{\volumedash}{%
  \makebox[0pt][l]{%
    \ooalign{\hfil\hphantom{$\m@th V$}\hfil\cr\kern0.08em--\hfil\cr}%
  }%
}

% Custom enumeration scheme
\newcommand\setItemnumber[1]{\setcounter{enumi}{\numexpr#1-1\relax}}

% Format margins
\addtolength{\oddsidemargin}{-.875in}
\addtolength{\evensidemargin}{-.875in}
\addtolength{\textwidth}{1.75in}
\addtolength{\topmargin}{-.875in}
\addtolength{\textheight}{1.75in}

\title{4500/6500 HW MATLAB Code}
\author{Daniel Agramonte}
\date{December 10 2020}

\begin{document}

\maketitle

\section*{MCHE\_6500\_NIST\_POLY.m}
\begin{lstlisting}[style=Matlab-editor]
function out = MCHE_6500_NIST_POLY(El,T_1,T_2,b)

% Convert to units that NIST uses
T_1 = T_1/1000;
T_2 = T_2/1000;

% Coefficients from NIST

% Argon
C1 = [20.786,2.825911*(10^-7),-1.464191*(10^-7),1.092131*(10^-8),...
-3.661371*(10^-8)];
M1 = 39.948*(10^-3);

% Oxygen
M2 = 2*15.999*(10^-3);

% Nitrogen
M3 = 28.0134*(10^-3);

% Helium
C4 = [20.78603,4.850638*(10^-10),-1.582916*(10^-10),...
    1.525102*(10^-11),3.196347*(10^-11)];
M4 = 4.0022602*(10^-3);

% Carbon Dioxide
M5 = 44.01*(10^-3);

R = 8.31446261815324;

% C_p function
C_p = @(t,C) C(1)+C(2)*t+C(3)*(t.^2)+C(4)*(t.^3)+(C(5)./(t.^2)); 

% C_p function divided by t
C_p_t = @(t,C) C(1)./t+C(2)+C(3)*t+C(4)*(t.^2)+(C(5)./(t.^3));

if El == "Ar"
    delta_h = (10^3)*(integral(@(t) C_p(t,C1),T_1,T_2))/M1;

    delta_u = (10^3)*(integral(@(t) C_p(t,C1)-(R),T_1,T_2))/M1;
    
    delta_s_1 = (integral(@(t) C_p_t(t,C1),T_1,T_2))/M1;
    
    C_P = C_p(T_1,C1)/M1;
    
    C_v = C_p(T_1,C1)/M1-8.3145/M1;
elseif El == "O"
    delta_h = (10^3)*(integral(@(t) C_p(t,C2(t)),T_1,T_2))/M2;

    delta_u = (10^3)*(integral(@(t) C_p(t,C2(t))-(R),T_1,T_2))/M2;
    
    delta_s_1 = (integral(@(t) C_p_t(t,C2(t)),T_1,T_2))/M2;
    
    C_P = C_p(T_1,C2(T_1))/M2;
    
    C_v = C_p(T_1,C2(T_1))/M2-8.3145/M2;
elseif El == "N"
    delta_h = (10^3)*(integral(@(t) C_p(t,C3(t)),T_1,T_2))/M3;

    delta_u = (10^3)*(integral(@(t) C_p(t,C3(t))-(R),T_1,T_2))/M3;
    
    delta_s_1 = (integral(@(t) C_p_t(t,C3(t)),T_1,T_2))/M3;
    
    C_P = C_p(T_1,C3(T_1))/M3;
    
    C_v = C_p(T_1,C3(T_1))/M3-8.3145/M3;
elseif El == "He"
    delta_h = (10^3)*(integral(@(t) C_p(t,C4),T_1,T_2))/M4;
    
    delta_u = (10^3)*(integral(@(t) C_p(t,C4)-(R),T_1,T_2))/M4;

    delta_s_1 = (integral(@(t) C_p_t(t,C4),T_1,T_2))/M4;
    
    C_P = C_p(T_1,C4)/M4;
    
    C_v = C_p(T_1,C4)/M4-8.3145/M4;
elseif El == "Air"
    delta_h = (10^3)*(0.01*((integral(@(t) C_p(t,C1),T_1,T_2))/M1)+...
        0.21*((integral(@(t) C_p(t,C2(t)),T_1,T_2))/M2)+...
        0.78*((integral(@(t) C_p(t,C3(t)),T_1,T_2))/M3));

    delta_s_1 = 0.01*((integral(@(t) C_p_t(t,C1),T_1,T_2))/M1)+...
        0.21*((integral(@(t) C_p_t(t,C2(t)),T_1,T_2))/M2)+...
        0.78*((integral(@(t) C_p_t(t,C3(t)),T_1,T_2))/M3);
    
    C_P = (0.01*(C_p(T_1,C1)/M1)+0.21*(C_p(T_1,C2(T_1))/M2)+...
        0.78*(C_p(T_1,C3(T_1))/M3));
    
    C_v = (0.01*((C_p(T_1,C1)-R)/M1)+0.21*...
        ((C_p(T_1,C2(T_1))-(R))/M2)+0.78*((C_p(T_1,C3(T_1))-...
        (R))/M3));
elseif El == "CO2"
    delta_h = (10^3)*(integral(@(t) C_p(t,C5(t)),T_1,T_2))/M5;

    delta_u = (10^3)*(integral(@(t) C_p(t,C5(t))-(R),T_1,T_2))/M5;
    
    delta_s_1 = (integral(@(t) C_p_t(t,C5(t)),T_1,T_2))/M5;
    
    C_P = C_p(T_1,C5(T_1))/M5;
    
    C_v = C_p(T_1,C5(T_1))/M5-8.3145/M5;
end

if b == 1
    out = delta_s_1;
elseif b == 0
    out = delta_h;
elseif b == 2
    out = delta_u;
elseif b == 3
    out = C_P;
elseif b == 4
    out = C_v;
elseif b == 5
    out = C_P/C_v; 
end

function out = C2(t)

if t<0.7
    % 100K-700K
    out = [31.32234,-20.23531,57.86644,-36.50624,-0.007374];
    
elseif t<2
    % 700K-2000K
    out = [30.03235,8.772972,-3.988133,0.788313,-0.741599];
else
    % 2000K-6000K
    out =[20.91111,10.72071,-2.020498,0.146449,9.245722];
end

function out = C3(t)

if t<0.5
    % 100K-500K
    out = [28.98641,1.853978,-9.647459,16.63537,0.000117];
elseif t<2
    % 500K-2000K
    out = [19.50583,19.88705,-8.598535,1.369784,0.527601];
else
    % 2000K-6000K
    out = [35.51872,1.128728,-0.196103,0.014662,-4.552760];
end

function out = C5(t)

if t<1.2
    % 298K-1200K
    out = [24.99735,55.18696,-33.69137,7.948387,-0.136638];
else
    % 1200K-6000K
    out = [58.16639,2.720074,-0.492289,0.038844,-6.447293];
end
\end{lstlisting}

\section*{MCHE\_6500\_Stag.m}
\begin{lstlisting}[style=Matlab-editor]
function out = MCHE_6500_Stag(x,gamma,f)

% T/T_0 along an isentropic flow path 
t_ratio = 1/(1+0.5*(gamma-1).*x.^2);

% P/P_0 along an isentropic flow path
p_ratio = (1+0.5*(gamma-1).*x.^2).^(gamma./(1-gamma));

% rho/rho_0 along an isentropic flow path
rho_ratio = (1+0.5*(gamma-1).*x.^2).^(1./(1-gamma));

% M in terms of gamma and T/T_0 and is valid along an isentropic flow path
M_t = (2*((gamma-1).^-1)*((x.^-1)-1)).^0.5;

% M in terms of gamma and P/P_0 and is valid along an isentropic flow path
M_p = (2*((gamma-1).^-1)*((x^((1-gamma)./gamma))-1))^0.5;

% M in terms of gamma and rho/rho_0 and is valid along an isentropic 
% flow path
M_rho = (2*((gamma-1).^-1)*((x^(1-gamma))-1))^0.5;

% Area ratio s.t. when the flow is choked, the mass flow rate is maximized
A_ratio = (x.^-1).*(2*((gamma+1).^-1)*(1+0.5*(gamma-1).*x.^2)).^...
    ((0.5*(gamma+1))./(gamma-1));

if f == 1
    out = t_ratio;
elseif f == 2
    out = p_ratio;
elseif f == 3
    out = rho_ratio;
elseif f == 4
    out = M_t;
elseif f == 5
    out = M_p;
elseif f == 6
    out = M_rho;
elseif f == 7
    out = A_ratio;
end
\end{lstlisting}

\newpage

\section*{MCHE\_6500\_Normal\_Shock.m}
\begin{lstlisting}[style=Matlab-editor]
function out = MCHE_6500_Normal_Shock(M1,gamma,Options)

Options = convertStringsToChars(Options);

if isequal(Options,'M2')
    out = ((M1^2+2*((gamma-1)^-1))/(2*gamma*((gamma-1)^-1)*M1^2-1))^0.5;
elseif isequal(Options,'P2')
    out = 2*gamma*((gamma+1)^-1)*M1^2-(gamma-1)/(gamma+1);
elseif isequal(Options,'T2')
    out = ((1+0.5*((gamma-1))*M1^2)*(gamma*M1^2-0.5*(gamma-1)))/...
        ((0.5*(gamma+1)*M1)^2);
elseif isequal(Options,'P02')
    out = (((0.5*(gamma+1)*M1^2)*((1+0.5*(gamma-1)*M1^2)...
        ^-1))^(gamma/(gamma-1)))/((2*gamma*((gamma+1)^-1)*M1^2-...
        (gamma-1)*((gamma+1)^-1))^(1/(gamma-1)));
elseif isequal(Options,'T02')
    out = 1;
    disp('You big dumb idiot! T_0,1=T_0,2 bc the process is adiabatic')
else
    disp('Not a valid Option!')
    out = 69;
end
end


\end{lstlisting}

\newpage

\section*{MCHE\_6500\_Oblique\_Shock.m}
\begin{lstlisting}[style=Matlab-editor]
function out = MCHE_6500_Oblique_Shock(M1,gamma,beta,Options)

Options = convertStringsToChars(Options);

if isequal(Options,'M2n')
    out = (((M1*sind(beta))^2+2*((gamma-1)^-1))/(2*gamma*((gamma-1)^-1)*...
        (M1*sind(beta))^2-1))^0.5;
elseif isequal(Options,'P2')
    out = 2*gamma*((gamma+1)^-1)*((M1*sind(beta)))^2-(gamma-1)/(gamma+1);
elseif isequal(Options,'T2')
    out = ((2*gamma*(((M1*sind(beta)))^2)-(gamma-1))*((gamma-1)*...
        ((M1*sind(beta))^2)+2))/(((gamma+1)^2)*((M1*sind(beta))^2));
elseif isequal(Options,'P02')
    out = ((((gamma+1)*((M1*sind(beta))^2))/((gamma-1)*((M1*sind(beta))...
        ^2)+2))^(gamma/(gamma-1)))*(((gamma+1)/(2*gamma*((M1*sind(beta))...
        ^2)-(gamma-1)))^(1/(gamma-1)));
elseif isequal(Options,'tantheta')
    out = (2*cotd(beta)*((M1*sind(beta))^2-1))/((M1^2)*...
    (gamma+cosd(2*beta))+2);
elseif isequal(Options,'T02')
    out = 1;
    disp('You big dumb idiot! T_0,1=T_0,2 bc the process is adiabatic')
else
    disp('Not a valid Option!')
    out = 69;
end
end


\end{lstlisting}

\newpage

\section*{MCHE\_6500\_Fanno.m}
\begin{lstlisting}[style=Matlab-editor]
function out = MCHE_6500_Fanno(M,gamma,Options)

Options = convertStringsToChars(Options);

if isequal(Options,'fL/d')
    out = (1-(M^2))/(gamma*(M^2))+((gamma+1)/(2*gamma))*log((M^2)/...
          ((2/(gamma+1))*(1+((gamma-1)/2)*(M^2))));
elseif isequal(Options,'T/T*')
    out = 0.5*(gamma+1)/(1+0.5*(gamma-1)*(M^2));
elseif isequal(Options,'v/v*')
    out = ((0.5*(gamma+1)*(M^2))/(1+0.5*(gamma-1)*(M^2)))^0.5;
elseif isequal(Options,'rho/rho*')
    out = ((0.5*(gamma+1)*(M^2))/(1+0.5*(gamma-1)*(M^2)))^-0.5;
elseif isequal(Options,'P/P*')
    out = ((0.5*(gamma+1)*(1+0.5*(gamma-1)*M^2)^-1)^0.5)/M;
elseif isequal(Options,'P0/P0*')
    out = (M^-1)*(2*((gamma+1)^-1)*(1+0.5*(gamma-1)*M^2))^...
          ((gamma+1)/(2*(gamma-1)));
elseif isequal(Options,'T0/T0*')
    out = 1;
    disp('You big dumb idiot! T_0/T_0*=constant bc the process is adiabatic')
else
    disp('Not a valid Option!')
    out = 69;
end
end




\end{lstlisting}

\section*{MCHE\_6500\_Rayleigh.m}
\begin{lstlisting}[style=Matlab-editor]
function out = MCHE_6500_Rayleigh(M,gamma,Options)

Options = convertStringsToChars(Options);

if isequal(Options,'T/T*')
    out = (M*(1+gamma)/(1+gamma*(M^2)))^2;
elseif isequal(Options,'v/v*')
    out = (M^2)*((1+gamma)/(1+gamma*(M^2)));
elseif isequal(Options,'rho/rho*')
    out = ((M^2)*((1+gamma)/(1+gamma*(M^2))))^-1;
elseif isequal(Options,'P/P*')
    out = (1+gamma)/(1+gamma*(M^2));
elseif isequal(Options,'P0/P0*')
    out = ((1+gamma)/(1+gamma*(M^2)))*(((2/(gamma+1))*(1+0.5*(gamma-1)*...
          (M^2)))^(gamma/(gamma-1)));
elseif isequal(Options,'T0/T0*')
    out = (2*(gamma+1)*(M^2)*(1+0.5*(gamma-1)*(M^2)))/((1+gamma*(M^2))^2);
else
    disp('Not a valid Option!')
    out = 69;
end
end

\end{lstlisting}

\section*{MCHE\_6500\_Rayleigh.m}
\begin{lstlisting}[style=Matlab-editor]
function [Bweak,Bstrong] = MCHE_6500_Collars_Method(M1,theta,gamma)
% Calculate coefficients on poly
A = M1.^2 - 1;
B = 0.5*(gamma+1)*M1^4*tand(theta);
C = (1 + 0.5*(gamma+1)*M1^2)*tand(theta);

% Initial value iteration will start from
X0 = A^0.5;
Xn = X0;
delt = 1;
while delt > 0.000009
    % Iterative process
    Xnew = (A-B*(Xn+C)^(-1))^0.5;
    delt = Xn-Xnew;
    Xn = Xnew;
end
XI = Xn;
XII = 0.5*(-(XI+C)+((XI+C)*(C-3*XI)+4*A)^.5);
Bweak = acotd(XI);
Bstrong = acotd(XII);
\end{lstlisting}

\section*{MCHE\_6500\_Rayleigh.m}
\begin{lstlisting}[style=Matlab-editor]
function T = MCHE_6500_Standard_Air(h)

if h<11
    T = 288.2-(71.7/11)*h;
elseif h<20.1
    T = 216.5;
elseif h<32.2
    T = 216.5+(12/12.1)*(h-20.1);
elseif h<47.3
    T = 228.5+(42/15.1)*(h-32.2);
elseif h<52.4
    T = 270.5;
elseif h<61.6
    T = 270.5-(18/9.2)*(h-52.4);
elseif h<80
    T = 252.5-(72/18.4)*(h-61.6);
else
    T = 180.5;
end

    
\end{lstlisting}

\end{document}
